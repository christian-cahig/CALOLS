\documentclass[journal, a4paper]{IEEEtran}
%\IEEEoverridecommandlockouts
% The preceding line is only needed to identify funding in the first footnote. If that is unneeded, please comment it out.

\usepackage{cite}
\usepackage{amsmath,amssymb,amsfonts}
\usepackage{algorithm}
\usepackage{algorithmic}
\usepackage{bm}
\usepackage{graphicx}
\usepackage{textcomp}
\usepackage[dvipsnames]{xcolor}
\usepackage{systeme}
%\usepackage{IEEEtrantools}

%% Custom Additions
\usepackage{array}
\usepackage{caption,subcaption}
\usepackage{datetime}
\usepackage{mathtools}
\usepackage{hyperref}
\usepackage{cleveref}
\usepackage{fontawesome5}
\usepackage{lipsum}
\usepackage{mleftright}
\usepackage{optidef}
\usepackage{physics}
\usepackage{xfrac}
%\usepackage[utf8]{inputenc}
%\usepackage[english]{babel}
%\usepackage{float}

% BibTeX
\def\BibTeX{{\rm B\kern-.05em{\sc i\kern-.025em b}\kern-.08em
		T\kern-.1667em\lower.7ex\hbox{E}\kern-.125emX}}

% Theorems
\newtheorem{theorem}{Theorem}[section]

% Floor and ceiling functions
\DeclarePairedDelimiter\ceil{\lceil}{\rceil}
\DeclarePairedDelimiter\floor{\lfloor}{\rfloor}

%% Front matter stuff
\title{A Consumer Appliance-Level Framework\\for Optimized Load Shedding}

%\author{
%	\IEEEauthorblockN{
%		Christian Y. Cahig\IEEEauthorrefmark{1}, Abdul Aziz G. Mabaning\IEEEauthorrefmark{2} \\
%	}
%	\IEEEauthorblockA{
%		\textit{Department of Electrical Engineering and Technology, College of Engineering and Technology}\\
%		\textit{Mindanao State University - Iligan Institute of Technology}\\
%		\IEEEauthorrefmark{1}chriscahig@gmail.com,
%		\IEEEauthorrefmark{2}mraaguevarra@gmail.com
%	}
%}

\author{
	Christian.~Y.~Cahig \textsuperscript{\faIcon[regular]{envelope}}, Abdul~Aziz~G.~Mabaning%
\thanks{
	The authors are with the Department of Electrical Engineering and Technology,
	Mindanao State University - Iligan Institute of Technology,
	Iligan City, Philippines.
}%
\thanks{\faIcon[regular]{envelope} {\color{blue}christian.cahig@g.msuiit.edu.ph}}%
\thanks{\faIcon{github} {\color{blue}https://github.com/christian-cahig/CALOLS}}
}

%% Document
\begin{document}

\maketitle

%\noindent
%\textit{Updated as of {\today\ \currenttime}.}\\

\begin{abstract}
\lipsum[17]
\end{abstract}

\begin{IEEEkeywords}
Optimization, power systems, optimal load shedding.
\end{IEEEkeywords}

\section{Introduction}
\label{sec: Introduction}

Load shedding is a course of action a utility company can take
when the balance between power supply and demand is at risk \cite{Jabian2018},
or when the system voltage or frequency exceeds its allowable lower limit for more than a minute \cite{GMC2016}.
Compared to investing on the installation of generating capacities to curb threats on the supply-demand balance,
load shedding is more economical and realistic in short-term considerations.

Since this strategy directly deals with the customers, load shedding has to be an optimized process.
As such, one can find numerous works in literature
(\textit{e.g.},
fast load shedding \cite{Wester2014},
clustering-based load shedding \cite{Potel2019}, and
load shedding during disaster events \cite{Babaei2020})
on feeder-level implementations.
In this setting, the smallest ``sheddable" unit of load is the total load connected to a feeder.
This coarse treatment is prone to excessive shedding of load and an unoptimized utilization of available supply.
Accordingly, we see works on residence-level implementation:
\textit{e.g.},
using smart meters, single-board computers, server monitoring \cite{Bhattacherjee2019},
detecting rate-of-change of frequency \cite{Sigrist2015},
and using a mobile distribution-level PMU \cite{Yao2020}.

Moving forward in this trend, \cite{Jabian2020} proposed a shedding framework where the smallest ``sheddable" unit of load is a consumer appliance.
This granular scheme addresses the concern of excessive load dropping in a feeder-level implementation,
as well as improves the operation of a residence-level setup.
Keeping both the utility and the customer in the framework means both parties can make the most of the situation in their perspectives.
By having the ability to prioritize appliances according to personal judgment, a consumer can avoid total interruption of comfort and/or productivity within his/her premises.
On the other hand, the utility can maximize the utilization of available power supply.

This work is a reimplementation of the framework originally proposed by \cite{Jabian2020}
and is part of the curricular requirements of an academic course on Optimization in Power Systems.
Although we are building on the basic formulation in \cite{Jabian2020},
this present work differs in a number of ways.
First, we critique the original work and emphasize some methodological rooms for improvement.
Second, as the dataset used in the original study is undisclosed,
we develop synthetic datasets based on the well-known open-sourced test cases
(\textit{e.g.}, IEEE 37-Node Test Feeder \cite{Kersting2001}).
Third, we consider different solution strategies,
in addition to the heuristics-based approach in \cite{Jabian2020},
for the core optimization problem.
% TO DO: Update list of contributions.
Lastly, to promote reproducibility, we make our data files, source codes, and documentation available to the public.

The remainder of this paper proceeds as follows.
Section \ref{sec: Basic Problem Formulation} presents the consumer appliance-level framework
and formulates the core optimization problem.
In Section \ref{sec: A Critique of the Original Work}, we discuss a critique of the original work of \cite{Jabian2020}
and propose some methods to address the identified issues.
Section \ref{sec: Datasets} details the procedure, analyses, and benchmarks for the datasets developed.
% TO DO: Update overview of sections.
Section \ref{sec: Conclusion} concludes the work.

\section{Basic Problem Formulation}
\label{sec: Basic Problem Formulation}

% TO DO: Section introduction

\lipsum[8]

\subsection{System Configuration}
\label{subsec: I. System Configuration}

\subsection{Nomination of Appliances for Load Shedding}
\label{subsec: I. Nomination of Appliances for Load Shedding}

\subsection{Consumer-side Optimization}
\label{subsec: I. Consumer-side Optimization}

\subsection{Utility-side Optimization}
\label{subsecL I. Utility-side Optimization}

\section{A Critique of the Original Work}
\label{sec: A Critique of the Original Work}

% TO DO: Section introduction

\lipsum[48]

\subsection{On the Cascading Structure of Consumer-side Optimization}
\label{subsec: II. On the Cascading Structure of Consumer-side Optimization}

\subsection{On the Rush to Heuristics}
\label{subsec: II. On the Rush to Heuristics}

\subsection{On Case Study Scenarios}
\label{subsec: II. On Case Study Scenarios}

\section{Datasets}
\label{sec: Datasets}

% TO DO: Section introduction

\lipsum[32]

\subsection{Preparation}
\label{subsec: III. Preparation}

\subsection{Quantitative Analyses}
\label{subsec: III. Quantitative Analyses}

\subsection{Benchmarks}
\label{subsec: III. Benchmarks}

\section{Conclusion}
\label{sec: Conclusion}

\lipsum[28]

%%%%%%%%%%%%%%%%%%%%%

% Acknolwedgement
%\vspace{40pt}
%\section*{Acknowledgment}

% References
%\vspace{40pt}
%\section*{References}
\bibliographystyle{IEEEtran}
%The argument is your BibTeX string definitions and bibliography database(s).
\bibliography{references}

\end{document}
